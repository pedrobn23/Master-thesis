% Tipo de documento (presentación)
\documentclass[usenames,dvipsnames]{beamer}

% Cargar el tema
\usetheme{metropolis}

% Configuración de LaTeX
\usepackage[spanish]{babel}
\usepackage[utf8]{inputenc}

\usepackage{listings}


\usepackage{graphicx,wrapfig,lipsum}

% Configuración básica del tema
\metroset{
  % tema oscuro ('dark') o claro ('light'). No tiene efecto al usar la
  % paleta de colores más adelante
  background=light,
  % 'none' para eliminar la diapositiva inicial de cada sección
  sectionpage=progressbar,
  % 'progressbar' o 'simple' para añadir una diapositiva inicial a cada subsección
  subsectionpage=none,
  % contador de página: 'none', 'counter' o 'fraction'
  numbering=none,
  % barra de progreso: 'none', 'head', 'frametitle' o 'foot'
  progressbar=frametitle,
  % fondo de los bloques estilo teorema: 'transparent' o 'fill'
  block=fill,
}


% Paleta de colores
\definecolor{accent}{RGB}{151, 186, 88}
\colorlet{darkaccent}{accent!70!black}
\definecolor{foreground}{RGB}{0, 0, 0}
\definecolor{background}{RGB}{255, 255, 255}

% Insertar los colores en el tema
\setbeamercolor{normal text}{fg=foreground, bg=background}
\setbeamercolor{alerted text}{fg=darkaccent, bg=background}
\setbeamercolor{example text}{fg=foreground, bg=background}
\setbeamercolor{frametitle}{fg=background, bg=accent}

\setbeamercolor{headtitle}{fg=background!70!accent,bg=accent!90!foreground}
\setbeamercolor{headnav}{fg=background,bg=accent!90!foreground}
\setbeamercolor{section in head/foot}{fg=background,bg=accent}

% 
\defbeamertemplate*{headline}{miniframes theme no subsection}{
  % Caja para mostrar título y autor encima de cada diapositiva
  % Nosotros no 
  %% \begin{beamercolorbox}[ht=2.5ex,dp=1.125ex,
  %%   leftskip=.3cm,rightskip=.3cm plus1fil]{headtitle}
  %%   {\usebeamerfont{title in head/foot}\insertshorttitle}
  %%   \hfill
  %%   \leavevmode{\usebeamerfont{author in head/foot}\insertshortauthor}
  %% \end{beamercolorbox}
  %% \begin{beamercolorbox}[colsep=1.5pt]{upper separation line head}
  %% \end{beamercolorbox}

  % Caja para mostrar navegación encima de cada diapositiva
  \begin{beamercolorbox}{headnav}
    \vskip2pt\insertnavigation{\paperwidth}\vskip2pt
  \end{beamercolorbox}
  \begin{beamercolorbox}[colsep=1.5pt]{lower separation line head}
  \end{beamercolorbox}
}

% eye-candy; sintasix más bonita
\newcommand{\seccion}[1]{\input{./sections/#1}}
\newcommand{\foo}{\hspace{-2.3pt}$\bullet$ \hspace{5pt}}
\newcommand{\R}{\mathbb{R}}
\newcommand{\D}{\mathbb{D}}
\newcommand{\T}{\mathbb{T}}
\newcommand{\N}{\mathbb{N}}
\newcommand{\K}{\mathbb{K}}
\newcommand{\re}{\operatorname{Re}}
\newcommand{\cow}{\overline{co}^{\omega^*}}

% Meta
\title{Cálculo lambda y Lógica Intuicionista}
\subtitle{Diferentes ópticas del mismo problema}
\date{26 de junio de 2021}
\institute{EsLibre 2021}
\author{Pedro Bonilla Nadal}
\titlegraphic{}
\newtheorem{lema}{Lema}
\newtheorem{teorema}{Teorema}
\newtheorem{corolario}{Corolario}
\newtheorem{proposicion}{Proposición}
\newtheorem{definicion}{Definición}


\begin{document}
\maketitle
\begin{frame}{Contenidos}
  \setbeamertemplate{section in toc}[sections numbered]
  \tableofcontents [hideallsubsections]
\end{frame}
\begin{frame}{Objetivos}
  \begin{enumerate}
  \item Dar las ideas que hay detrás de la visión formal de la teoría.
  \item Exponer resultados interesantes.
  \item Que todo el mundo se lleve algo a casa.
  \end{enumerate}
\end{frame}
\section{Cálculo no tipado}
\begin{frame}{Abstraer el problema de Computar}
  \begin{itemize}
  \item Los problemas del milenio de Hilbert.
  \item Solución de Turing inspirada en el cómputo ``con lápiz y papel''.
  \item Solución de Church inspirada en el concepto de sustitución.
  \end{itemize}
\end{frame}

\begin{frame}{Cálculo no Tipado}
\begin{definicion}
  The formulas of  $\lambda$-calculus are built via the BNF:
  $$A,B ::= x\ |\ (AB)\ |\ (\lambda x.A) .$$
  where $x$ denote any variable in $\mathcal{V}$.
\end{definicion}

\begin{example}
  Generic example
\end{example}

\end{frame}

\begin{frame}{Reducciónes}
  
\begin{table}[h]
  \begin{center}
    \begin{tabular}{|l|l|l|}
      \hline
      Name & Main rule & Equivalence \\
            \hline
      $\alpha$ & $ (\lambda x. M) =_\alpha \lambda y. M[x/y]$& Yes\\
      $\beta$ & $(\lambda x.M)N \to_\beta M[N/x]$& No\\
      $\eta$ & ${\displaystyle (\lambda x.Mx) \to_\eta M }$& No\\
      \hline
    \end{tabular}
  \end{center}
  \caption{\label{tab:reductions}Reducciones}
\end{table}

\end{frame}

\section{Cálculo lambda tipado}
\begin{frame}{¿Por qué querriamos tipos?}
  \begin{itemize}
  \item Tenemos 
  \end{itemize}
  
\end{frame}


\begin{frame}{Breve Presentación}

\end{frame}


\begin{frame}[Naturaleza del tipado]
  
\end{frame}


\begin{frame}{Números naturales}

\end{frame}



\section{Lógica Intuicionista}

\begin{frame}{Breve Presentación}

\end{frame}

\begin{frame}{Sistemas deductivos}

\end{frame}

\begin{frame}{Cálculo lambda tipado como Sistema deductivo}
  
\end{frame}
\section{Curry-Howard}

\begin{frame}{Curry-Howard}
  
\end{frame}

\begin{frame}{Más allá de Curry howard}

\end{frame}

\begin{frame}{Bibliografía y para saber más}

\end{frame}
\end{document}
