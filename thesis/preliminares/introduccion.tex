%*******************************************************
% Introducción
%*******************************************************

% \manualmark
% \markboth{\textsc{Introducción}}{\textsc{Introducción}} 
\chapter{Introduction}


The great revolution in mathematics, already started in the middle of the 19th century with the epsilon-delt formalism, is the formalization of mathematics. During this period, brilliant mathematicians like Weierstrass, Hilbert and most notably the group  Bourbaki put formalism at the core of mathematics. \\

Of all the areas that benefited from formalization, computation was probably the one that came out best. After the work of Church and Turing it became the mathematical element that would most seriously transform the world, like a steam engine of the contemporary era. This transformation was so radical that a person 50 years ago would not understand how to use the tools with which the population spends more than half of its waking hours nowadays.\\

The aim of this work is thus to strain the world-transforming idea of computation into an algebraic theory. This work  adopts a slow but steady pace, for before being able to strain one must first know how to use the straining tools, and then know in depth what one intends to strain.  \\

After the work of previous mathematicians such as Galois and Poincaré, and in sight of the new formalization and organization of mathematics, mathematicians such as MacLane, Elineberg, Cartan or Grothendiek began to form and expand what we know today as category theory. This theory effectively strained mathematical ideas in their purest form, and generate theories seemingly ubiquitous. After all, we study category as a knowledge condenser.\\

In the first part of this work we will talk about the theory of categories, explaining and introducing it. In the second part of the work, we will navigate in the concept of lambda calculus, as the first formalizing idea of computation. Both theories will be introduced with the point of view of a person who is interested in only one of them.\\

It is in the third part when the pairing happens. In particular the fourth chapter we raise all the bridges between closed Cartesian categories and computation with types. Finally, in chapter 5, we use ideas with the same flavor to prove a similar equivalence with Martin-L\"of's dependent typing. \\

Thus, this paper will deal with the intuition behind the formalism. We encourage the reader to try to gain in his reading an idea, not only formal but also intuitive, of what computation is, and what lies at the heart of this idea.  After all, in the immortal words of Karl Weierstrass \cite{duporcq1902compte}

\begin{quote}
... it is true that a mathematician who is not somewhat of a poet, will never be a perfect mathematician.\\
\end{quote}

\section*{Main goals and results achieved}
The initials goals of this master's thesis were:
\begin{enumerate}
\item Accomplishing a theoretical study of the foundations of category theory.
\item Accomplishing a theoretical study of the foundations of $\lambda$-calculus and typing theory.
\item Unifying both point of view after the works of Lambek \cite{lambek1988introduction} and Seely \cite{seely1984locally}.
\end{enumerate}

We consider that these three objectives have been accomplish successfully.\\

We also consider this work of a particular interest, because instead of introducing the concepts of category and typing in more focused on the later proof of equality, we prefer to work with the concepts in the form where their intuitive notion is clearer, in the preferred formulation that mathematicians whose objective is the study of each area and not its pairing with each other.



\endinput
