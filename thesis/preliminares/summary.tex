%*******************************************************
% Summary
%*******************************************************

\newpage



\chapter*{Summary}
\addcontentsline{toc}{chapter}{Summary}
\section*{Brief Summary}
Category theory is a very active field of research in algebra. Interested in this topic, we make in chapter 1 a general introduction of the structure, focusing in chapter 2 on the most interesting properties of this theory: universality and adjunctions. \\

Once this theory has been introduced, we proceed to the second interest of the work: typing. In chapter 3 we follow the historical path of the area, introducing the untyped lambda calculus to later introduce the concept of types. We conclude this chapter by presenting the curry howard isomorphism, which explains how we can understand a simply typed extended lambda calculus as propositional logic.\\

Finally, in chapter 4 we analyze the relations between the two theories, with Lambek's isomorphism. For this, we will first present both as a deductive system, and explain the functors necessary to realize an equivalence between the category of closed Cartesian categories and the simply typed lambda calculus.\\

Finally, we take advantage of these ideas in Chapter 5 to begin by explaining Martin-L\"of's theories of typing, which are intended to be a formalization of constructive mathematics. After this, we introduce the concept of locally closed Cartesian categories. Instrumental for the understanding of these will be the hyperdoctrines, which will present us with a structure of adjunctions that briefly summarizes the properties of the structure. Finally, we present the theories of martin lof as a category, and prove a new equivalence of categories between this new category and the category of locally closed Cartesian categories. \\

{keywords: Lambda-calculus, Category Theory, Hyperdoctrines, Martin-L\"of typing, Closed Cartesian Categories.} 


\newpage
\selectlanguage{spanish}
\section*{Resumen}
La teoría de categorías es un campo de investigación muy activo en álgebra. Interesados en este tema, hacemos en el capítulo 1 una introducción general de la estructura, centrándonos en el capítulo 2 en las propiedades más interesantes de esta teoría: la universalidad y las adjunciones. \\

Una vez introducida esta teoría, pasamos al segundo interés de la obra: la tipificación. En el capítulo 3 seguimos la trayectoria histórica del área, introduciendo el cálculo lambda no tipado para posteriormente introducir el concepto de tipos. Concluimos este capítulo presentando el isomorfismo de Curry Howard, que explica cómo podemos entender un cálculo lambda extendido simplemente tipado como lógica proposicional.\\

Finalmente, en el capítulo 4 analizamos las relaciones entre ambas teorías, con el isomorfismo de Lambek. Para ello, primero presentaremos ambas como un sistema deductivo, y explicaremos los funtores necesarios para realizar una equivalencia entre la categoría de categorías cartesianas cerradas y el cálculo lambda simplemente tipado.\\

 Finalmente, aprovechamos estas ideas en el capítulo 5 para empezar a explicar las teorías de tipificación de Martin-L\"of, que pretenden ser una formalización de la matemática constructiva. A continuación, introducimos el concepto de categorías cartesianas localmente cerradas. Para la comprensión de éstas serán instrumentales las hiperdoctrinas, que nos presentarán una estructura de adjuntos que resume brevemente las propiedades de la estructura. Finalmente, presentamos las teorías de martin lof como una categoría, y demostramos una nueva equivalencia de categorías entre esta nueva categoría y la categoría de categorías cartesianas localmente cerradas. \\

{palabras clave:Lambda-calculo, Teoria de categor\'ias, Hyperdoctrinas, Tipado de Martin-L\"of, Categorias cartesianas cerradas.} 


\selectlanguage{english}


% \endinput
