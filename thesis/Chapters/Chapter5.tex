
\chapter{Locally Cartesian closed categories}
\label{chap:5}
\thispagestyle{empty}

 In this we use hyperdoctrines as a tool to better understand closed Cartesian categories and relate them with a new typing theory: Martin-Lof type theory.

In this chapter we introduce the notions of \cite{seely1984locally}. We continue our categorical presentation of the tip theory with the introduction of Martin Lof's type theory. This type theory, also known as intuitionistic type theory, is initially proposed as a foundation of mathematics. While I remember the sections ... in which we explained the relationship between the simply typed lambda calculus devised by curch and curry and propositional logic, this new type theory is now intended to be a complete foundation for mathematics, including first-order logic (always with no third excluded).

This relationship with first-order logic arises from the hand of dependent typing. That is, in Martin Lof's type theory there is a formal structure for generating types dependent on other types. This is used (without needing to know the formalism) nowadays continuously in programming languages. 

The most interesting part for us will be to study the new structures that arise in the theory of categories to understand this new typing. In particular, we will study local closed Cartesian category structures through hyperdoctrines. 

The main sources for this chapter are {\color{red} TODO}



\section{Martin-Löf type theory}