
\chapter{Universality, Adjoints and Monads}

\section{Universality}
\subsection{Definition}
In this section we present the concept of universality. This concept is behind lots of mathematical properties. Intuitively, universality is an efficient way of expressing an one-to-one correspondence between arrows of different categories. This one-to-one relationship is usually expressed via ''given an arrow  $y$ it exists one and only one  arrow $x$ such that <insert your favorite universality property''.\\

Prior to the formal definition, we shall introduce an example. Probably the first contact that any mathematician has with universality is when we first try to define a function  $f:\mathbb R \to \mathbb R^2$. We quickly understand that defining such a function is equivalent to define two $g,h: \mathbb R \to \mathbb R$( we further explain the product in \ref{prod-univ}). This uniqueness is the flavor that attempts to capture the concept of universality. Other examples of unique existence are those that occur in quotient groups or in bases of vector spaces.  This example will be further formalized after the definition.\\

\begin{definition}
  Let $S: D \to C$ be a functor and $c \in Ob(C)$, an \emph{universal arrow}  from $c$ to $S$ is a pair $(d,u)$ with $d\in Ob(D), u:c \to Sd \in Ar(C)$, such that for every $(e,f)$ with $e\in Ob(D)$  and $f:c\to Sd$ there exists an unique $f':r,d\in Ar(D)$ such that $u\circ Sf' = f$.

\end{definition}

In a diagram:

\[
  \begin{tikzcd}
      c\arrow[r, "u"] \arrow[rd, "f"]      & Sd \arrow[d, "Sf'", dashed]& d\arrow[d,"f'", dashed]\\
       &Se& e 
    \end{tikzcd}
  \]

  Note that an universal arrow $(d,u)$ induces the unique existence of an arrow in $D$, but with interesting properties via it relationship with $S$. Whenever possible to provide an universal arrow we will only define the functor $S:D\to C$ and the arrow $u:c\to Sr$, letting all other information be deduced from the context.\\


  Lets formalize the prior examples and provide some more:

  \begin{example}\ 
    \begin{itemize}
    \item Quotient Group:

      From this property alone the three isomorphism theorems can be deduced. Therefore, we only have to prove this result to have the full power of these theorems in any context (e.g. Rings, K-Algebra, or Topological spaces).
    \item Product in $\R$:\label{example product}
    \item Vector Space Bases:
    \item Haskell Type void (or some other cool example):
    \end{itemize}
  \end{example}
  
  Lastly, we will provide a characterization of universality:
  \begin{proposition}
      Let $S: D \to C$ be a functor and $u:c\to Sr\in Ar(C)$. Then $u$ is an universal arrow if and, only if, the function $\varphi:\hom_D(r,\cdot)\to \hom(c,S\cdot)$ such that $\varphi(d)(f)= Sf\circ u$ for all $f\in \hom_D(r,d)$ is a natural bijection. Conversely, every natural bijection is uniquely determined by an universal arrow $u:c\to Sr$. 
  \end{proposition}
  \begin{proof}
    Lets start by supposing that $u$ is universal. Then for $\varphi$ to be universal the diagram 
    \[
      \begin{tikzcd}
        \hom_D(r,d)\arrow[swap]{d}{\hom_D(r,g)}\arrow{r}{\varphi(d)} & \hom_C(r,Sd)\arrow{d}{\hom_C(r,Sg)}\\
        \hom_D(r,d')\arrow{r}{\varphi(d')} & \hom_C(r,Sd')
      \end{tikzcd}
    \]

    should commute for all $g\in Ar(D)$. As this is a diagram in the category of sets, we can check the commutativity by checking it element wise. For any $f \in\hom_D(r,d)$:
    \[
      \begin{tikzcd}
        f\arrow[swap]{d}{\hom_D(r,g)}\arrow{r}{\varphi(d)} & Sf\circ u \arrow{d}{\hom_C(r,Sg)}\\
        g\circ f \arrow{r}{\varphi(d')} & S(g\circ f) \circ u = Sg\circ Sf \circ u
      \end{tikzcd}
    \]

    So the diagram commutes, and $\varphi$ is natural. The bijectivity follows from the definition of $u$ being universal.\\

    Lets consider now that $\varphi$ is a natural bijection. We will define $u := \varphi(r)(1_r)$ and check that $(r,u)$ is an universal arrow. As $\varphi$ is natural we have that:

    \[
      \begin{tikzcd}
        \hom_D(r,r)\arrow[swap]{d}{\hom_D(r,g)}\arrow{r}{\varphi(r)} & \hom_C(r,Sr)\arrow{d}{\hom_C(r,Sg)}\\
        \hom_D(r,d)\arrow{r}{\varphi(d)} & \hom_C(r,Sd)
      \end{tikzcd}
    \]

    Writing the diagram for the element $1_r\in\hom_D(r,r)$ and any $d\in Ob(D), g:r\to d\in \hom_D(r,d)$:

    \[
      \begin{tikzcd}
        1_r \arrow[swap]{d}{\hom_D(r,g)}\arrow{r}{\varphi(r)} & u \arrow{d}{\hom_C(r,Sg)}\\
        g\arrow{r}{\varphi(d)} & \varphi(d)(g)= Sg\circ u
      \end{tikzcd}
    \]

    and since $\varphi$ is a bijection, for every $f\in \hom_C(r,Sd)$ there is an unique function $f' = \varphi(d)^{-1}(f)$ such that $Sf'\circ u = f$, thus being $u$ universal.

\end{proof}
From this theorem there is a definition that arises:

\begin{definition}
  Let $D$ be a category with small home-sets and let $F:D\to C$ be a functor. A representation of a functor $K:D\to Set$ is a pair $(r,\varphi)$ with $r \in Ob(D)$ and $ \varphi$ a natural isomorphism  such that
  $$D(r,\cdot) \cong_{\varphi} F.$$
  A functor is representable if it has a representation.
\end{definition}

Note that therefore a universal arrow induces a natural isomorphism $D(r,d)\cong C(c,Sd)$ and this induces a representation of the functor $C(c,S\cdot): D\to Set$. 

\subsection{Yoneda's lemma}
Yoneda's lemma is one of the main results of category theory. This results is due to japanese professor Nobuo Yoneda. We know about Yoneda's life thanks to the elegy that was written by Yoshiki Kinoshita\cite{yonedaLife}. Yoneda was born in Japan in 1930, and received his doctorate in mathematics from Tokyo University in 1952. He was a reviewer for international mathematical journals. In addition to his contributions to the field of mathematics, he also devoted his research to computer science.\\

{\color{red} TODO: further schematize this subsection when is finished.}


Mac Lane\cite{mac2013categories} assures the lemma first appeared in his private communication with Yoneda in 1954. With time, this result has became one of the most relevant one in Category Spaces.\\


{\color{red} (Maybe TODO)\footnote{To be honest, explain what a moduli problem is seems harder than explain the Yoneda lemma, therefore I will maybe write this after being more advanced in the text.}}The idea behind the Yoneda's lemma is better understood in the context of Moduli problems. 


Lets proceed to enunciate and proof the result.

\begin{theorem}\cite[Section 3.2]{mac2013categories}
  Let $D$ be a category with small home-sets, $F:D\to Set$ be a functor, and $r\in Ob(D)$. Then there is a bijection
  \begin{align*}
    \tau:Nat(\hom_D(r,\cdot), F\cdot) &\cong Fr\\
    \tau(\alpha:\hom_D(r,\cdot)\to F\cdot)& = \alpha(r)(1_r)
  \end{align*}
  Where $\tau$ is natural in $K$(as an object of $Set^{D}$) and in $r$.
\end{theorem}
\begin{proof}
  As $\alpha$ is a natural transformation we have that, for every $g:r\to d\in Ar(D)$: 
    \[
      \begin{tikzcd}
        \hom_D(r,r)\arrow[swap]{d}{\hom_D(r,g)}\arrow{r}{\alpha(r)} & Kr\arrow{d}{Kg}\\
        \hom_D(r,d)\arrow{r}{\alpha(d)} & Kd
      \end{tikzcd}
    \]
Writing $\alpha(r)(1_r) = u$ we have that:
    \[
      \begin{tikzcd}
        1_r\arrow[swap]{d}{\hom_D(r,g)}\arrow{r}{\alpha(r)} & u\arrow{d}{Kg}\\
        g\arrow{r}{\alpha(d)} & \alpha(d)(g)=Kg(u)
      \end{tikzcd}
    \]

    Therefore every natural transformation is uniquely identified by the value of $u$, therefore $\tau$ is injective. Moreover, for every $u$ in $Kr$, we can define a natural transformation following the previous diagram, therefore, $\tau$ is bijective.
\end{proof}

The first functor that we will want to apply this result is to the $\hom_{\cdot}$ functor. But this functor is a bifunctor, so to get the full result of this lemma applied to the full bifunctor we may restate this lemma to contravariant functors.

\begin{corollary}
  Let $D$ be a category with small home-sets, $F:D\to Set$ be a contravariant functor, and $r\in Ob(D)$. Then there is a bijection
  \begin{align*}
    \tau:Nat(\hom_D(\cdot,r), F\cdot) &\cong Fr\\
    \tau(\alpha:\hom_D(\cdot,r)\to F\cdot)& = \alpha(r)(1_r)
  \end{align*}
  
  Where $\tau$ is natural in $K$(as an object of $Set^{D}$) and in $r$.
\end{corollary}
\begin{proof}
We seek to use the Yoneda lemma in the functor $F':D^{op}\to Set$ induced by $F$. Then we have that: 
  \begin{align*}
    \tau:Nat(\hom_{D^{op}}(r,\cdot), F'\cdot) &\cong F'r\\
    \tau(\alpha:\hom_{D^{op}}(r,\cdot)\to F'\cdot)& = \alpha(r)(1_r)
  \end{align*}

  Taking into account that $F_{|Ob(D)}=F'_{|Ob(D^{op})}$, and that 
  $\hom_{D^{op}}(r,\cdot) = \hom_D(\cdot,r)$ we have the result.
\end{proof}

{\color{red} Is a bit messy after here, consider using subsections?}

As we have seen, the Yoneda lemma is a direct generalization of the moduli problem. In the same vein, yoneda's lemma is the generalisation of other problems/theorems in mathematics, most notably Cayley's lemma. It states:
\begin{displayquote}
``Any group is isomorphic to a subgroup of a symmetric group.'' 
\end{displayquote}

To understand this, take a groups $G$ seen as a single-object category, and name that object $e$. Then, the functor $\hom_G(e, \cdot): G \to Set$ can be seen as a group action \ref{group-action}. Then the Yoneda lemma states that:

$$Nat(\hom_G(e,\cdot), Hom_G(e,\cdot)) \cong_\varphi Hom_G(e,e).$$

Translating this result to group theory:
\begin{itemize}
\item Remember that $Hom_G(e,e)$ is the group $G$.
\item Every natural transformation is a equivariant map between G-sets.
\item This equivariant maps, forms an endormophism group under composition, being a subgroup of the group of permutations.
\item This natural isomorphism $\varphi$ define a group isomorphism.
\end{itemize}

So we have the isomorphism of groups that is stated in Cayley's Theorem.\\


Next,  \emph{Yoneda embedding}. To do this, we apply this result to the functor $h_a=\hom(a,\cdot)$ we get a bijection
$$Nat(h_a,h_b) \cong \hom_C(B,A),$$\\

{\color{red} TODO} Redactar Yoneba embedding.

\subsection{Some properties expressed in terms of universality}

  After the examples given, we will define a few constructions that are present in various parts. We will outline the notions of limit, pullback and product, and the dual notions of colimit, pushout and coproduct.\\

  The notions of product and pullback can be seen as particular cases of the notion of limit. We will therefore begin by defining this concept as an introductory step. In turn, to define limit we will introduce the concept of co-cone and the diagonal functor. \\

  \begin{definition}
    Let $C,J$ be categories. We can define the functor $\Delta_J: C \to C^J$ that maps $c$ to the functor from $J$ to $C$ that is constantly $c$, and maps every arrow to the identity $1_c$. 
  \end{definition}

  Whenever possible we will write only $\Delta$, and let the information of the category be deduced from context. $J$ is usually small and often finite. We can now consider a natural transformation $\tau: F \to \Delta c$. This can be represented as in the following diagram:
    \[
      \begin{tikzcd}
        Fx_j\arrow[swap]{dr}{\tau x_j}\arrow{rr}{F g} &
        & Fx_k\arrow{dl}{\tau x_k}\\
        & c&
      \end{tikzcd}
    \]

    commutes for every $g:x_j\to x_k\in Ar(j)$, for that reason, such natural transformation is usually called a co-cone. The dual notion is called cone and  is represented as:
    \[
      \begin{tikzcd}
        Fx_j\arrow{rr}{F g}&
        & Fx_k\\
        & c\arrow{ur}[swap]{\tau x_k}\arrow{ul}{\tau x_j} &
      \end{tikzcd}
    \]


    We can now define the concept of limit and colimit. We introduce first the concept of colimit. This definition is a basic definition of a universal arrow, only in a category of functors. Following this definition, we will define the limit as a dual concept.
  \begin{definition}
 A colimit is an object $r\in Ob(C)$ together with an universal arrow $u:F\to \Delta r \in Ar(C^J)$. The colimit is denoted by $$\lim_{\leftarrow} F = r = \colim F.$$
  \end{definition}

  The notation $\lim_{\leftarrow}$ is intuitive as we can see that in the colimit we have arrows to $F$. To represent this as a diagram, we have a co-cone $u\to \colim F$ such that for every other co-cone $\tau \to s$, it exist an unique $f$ such that the following commutes for every $x_j,x_k\in Ob(C)$:

\[
\begin{tikzcd}
{} & l& & \\
& \colim F   \arrow[dashed]{u}[description]{f} \\
x_j \arrow{ur}{u x_j} \arrow[bend left]{uur}{\tau x_j}\arrow[swap]{rr}{Fg} & & 
x_k \arrow[swap]{dr}{u x_k}\arrow[bend right,swap]{uul}{\tau x_k}\\
\end{tikzcd}
\]

  
Now, thanks to the duality of categories, we can define what a limit is in a very synthetic way:

\begin{definition}
  A limit is the dual concept of a colimit. It is denote as
  $$\lim_{\rightarrow} F = r = \lim F.$$
\end{definition}

A limit is represented by the following diagram:
\[
\begin{tikzcd}
{} & l
\arrow[bend right,swap]{ddl}{\tau x_j}
\arrow[bend left]{ddr}{\tau x_j} \arrow[dashed]{d}[description]{f}& & \\
& \lim F \arrow{dr}{u x_k} \arrow{dl}[swap]{u x_j} \\
x_j \arrow[swap]{rr}{Fg} & & 
x_k \\
\end{tikzcd}
\]

Analogously as in the colimit notation, in the limit we have arrows from $F$, and thus the notation $\lim_{\rightarrow}$. Let focus for a while now in the notion of limit, in particular of two of its specials cases: the product and the pullback. From these cases we are going to provide most examples. \\

We have already talked about the product of categories, and in example \ref{example product} we denote that in these type of construction there is some sort of universality involved. The product is a limit when $J$ is the 2-element discrete category, that is, when every functor from $J\to C$ is merely choosing to object of $C$. 

\begin{definition}\label{prod-univ}
  Let $C$ be a category, $J=\{0,1\}$ be the discrete category with two elements. The product $c_1\times c_2$ of two elements $c_0,c_1\in Ob(C)$ is the limit of the functor $F:J\to C$ such that $F0 = c_0, F1= c_1$.
\end{definition}

This construction means that providing an arrow to $c_0,c_1$ determines an unique arrow to $c_0\times c_1$. In this case, the arrows $u0, u1$ are usually called \emph{projections} and denoted by $\pi_0, \pi_1$. The notation is due to the product being a generalization of the cartesian product in the category $Set$. It is also sometimes note with $c_0 \Pi c_1$ with $\Pi$ being stardard for the sequence product. Some examples of product object in categories are:
\begin{example}\ 
\begin{itemize}
\item Product of Banach spaces:
\item Product of something in haskell.
\end{itemize}
\end{example}

Analogously, we can define the coproduct, on which instead of defining an arrow to $c_0, c_1$, we define an object \emph{from} $c_0,c_1$. 
\begin{definition}
  The coproduct is the dual definition of the product. It is denoted by $c_0 \sqcup c_1$.
\end{definition}
 In this notation $\sqcup$ denotes an inverted $\Pi$, with the meaning of being the dual notion of the product.
\begin{example}\ 
  \begin{itemize}
  \item Free product of groups
  \item Union of enumerate types.
  \end{itemize}
\end{example}



From my personal experience I have to say that I have seen more difficulties learning this notion rather than learning the utterly similar notion of product, probably because we are more used to think in terms of arrays rather than in terms of universality for the product. I think that thinking in terms of universality should be the way in to these concept.


After learning about the product and the coproduct, we will focus now in the notion of pullback and its dual, the pushout. We will first define the category

\[
  P = \begin{tikzcd}
    x\arrow{r}{f} & z & y\arrow[swap]{l}{g}
\end{tikzcd}
\]

Then we can define the pullback:

\begin{definition}
  Let $C$ be a category and $F:P\to C$ be a functor. Then the pullback of $Fx$ and $F_y$ denoted as $Fx\times_{Fz}Fy$ is the limit of the functor $F$.
\end{definition}

We can represent this structure in the following diagram. For any other object $q$ and arrows $f':q\to Fx,g':q\to Fy$ we have:

\[
\begin{tikzcd}
q
\arrow[bend left]{drr}{q_2}
\arrow[bend right,swap]{ddr}{q_1}
\arrow[dashed]{dr}[description]{u} & & \\
& Fx\times_{Fz}Fy \arrow{r}{p_2} \arrow{d}[swap]{p_1}
& Fy \arrow{d}{Fg} \\
& Fx \arrow[swap]{r}{Ff}
& Fz
\end{tikzcd}
\]

Analogously, we can define:

\[
  CoP = \begin{tikzcd}
    x & z\arrow{l}{f}\arrow[swap]{r}{g} & y
\end{tikzcd}
\]

and define:
\begin{definition}
  Let $C$ be a category and $F:CoP\to C$ be a functor. Then the pushout of $Fx$ and $F_y$ denoted as $Fx\sqcup_{Fz}Fy$ is the colimit of the functor $F$.
\end{definition}

\begin{example}
  \begin{itemize}
  \item Fiber bundles (pullback)
  \item Suppose that X, Y, and Z as above are sets, and that f : Z → X and g : Z → Y are set functions. The pushout of f and g is the disjoint union of X and Y, where elements sharing a common preimage (in Z)
   \item Seifert-Van-Kampen (Si tienes valor).
  \end{itemize}
\end{example}

We can regard that there are an evident similarity on the notation of the pullback/pushout and the one of product/coproduct. To understand these we have to consider the similarities of both construction. We are going to focus on the similarities of product and pullback.\\

In both case, the universal property consists of having an arrow to a generated object only if we have an arrow to each of its generators. In this line of reasoning, we can consider that the product is a pullback where we forget about the object $z$ and its arrows. One easy way to generate that case is to consider the construction when $Fz$ is a terminal object. In that case we have that the existence of $Ff$ and $Fg$ is a tautology, and we can only consider 

\[
\begin{tikzcd}
& q
\arrow[bend left]{dr}{q_2}
\arrow[bend right,swap]{dl}{q_1}
\arrow[dashed]{d}[description]{u} & & \\
 Fx  & Fx\times_{Fz}Fy \arrow{r}{\pi_1} \arrow{l}[swap]{\pi_0}& Fy \\
\end{tikzcd}
\]

having a product structure.  One can proceed to consider an analogous consideration with pushout and coproduct.
\section{Adjoints}


\begin{itemize}
\item Def
\item Note that every adjuntion raise a universal arrow
\end{itemize}

\section{Monad}