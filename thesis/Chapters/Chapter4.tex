
\chapter{Curry-Howard-Lambeck bijection}
In these chapter we raise bridges between Category Theory, Lambda calculus, and Propositional logic. The main sources for this chapter were  \cite{lambek1988introduction} and \cite[Chapter 6]{selinger2008lecture}.\\

\section{Deduction systems}

\subsection{First Definitions}
\subsubsection{Deduction systems}
Having introduced the notion lambda calculus, category theory and propositional intuitionistic calculus, is time to have a step back, in other to get the full picture.

\begin{definition}
  A \emph{Deductive system} $D$ is a graph  with identity and composition of arrows.
\end{definition}

We can reformulated the previous logic with this optic. Prior to that we will make an illustrative example on how is this made. All the properties aforementioned (and paired) show that the requirements of typing/intuitionistics calculus can be expressed in just requirements for the arrows. For example, 
\begin{example}
\begin{enumerate}
\item Deduction rule $\to_1$: There is an arrow $$\varepsilon_{A,B}: [B\land (B \to A)] \to A.$$
\item From deduction rule $\to_2$ it can be deduced:
  \begin{align*}
    \Gamma,\{\} \vdash & F= (C\land B)\to A;\\
    \Gamma, \{(C\land B)\} \vdash & A;\\
    \Gamma, \{C, B\} \vdash & A;\\
    \Gamma, \{C\} \vdash & B\to A;\\
    \Gamma, \{\} \vdash & F'=C \to (B \to A).
  \end{align*}
  That is provided an arrow $h: C\land B \to A$ there is an induced arrow
  $$h^*:C \to [A\to B].$$
\end{enumerate}
\end{example}

We can unpack a bit from these notions. To begin with, we can check that the second example include an idea, much needed, in the relationship between assumption and deduction systems. We can view each assumption of truth as a new deduction system. This also seems logical, since an assumption $\Gamma = \{A_i : i=1,...,n\}$  with $A_i$ formulas can be seen as an existence of an arrow $g_i: \top \to A_i$.\\

We see that in our derivations we sometimes need to change the truth-assumption $\Gamma$. This is not mandatory, but it is useful to avoid including a large number of rules. However we do that in a way such that we start from a given $\Gamma$ and return to it at the end of the derivation. Thus knowing that if a $\Gamma$ derives the formula $F$, then it also derives the formula $F'$ and therefore there is an arrow in our deductive system. Similar reflections can be made with the rest of the rules discussed in the previous sections, and of course with the simply typed lambda calculus. \\



\subsubsection{Logical Systems as Deduction systems}
In  the fashion of previous definitions we can reformulate the logical systems as deduction systems.\\
\begin{definition}
  A conjunction calculus is a deductive system $D$ where there is a specified object $\top$ and an operation $\land: Ob(D)\times Ob(D) \to Ob(D)$ such that:
  \begin{enumerate}
  \item For each $A\in Ob(D)$ exists $O_A:A\to \top$.
  \item For each $A_1,A_2\in Ob(D)$ exists arrows $\pi_i^{A_1,A_2}: A_1 \land A_2 \to A_i$ for $i=1,2$.
  \item Let $f:C\to A$ and $g:C\to B$. Then $\langle f,g\rangle C\to A\land B$.
  \end{enumerate}
\end{definition}

Note that $\pi_i^{A_1,A_2}$ depend on both $A_1,A_2$. in the absence of ambiguity we will denote this arrow by $\pi_i$.

\begin{definition} \label{def:positivecalculus}
  A positive intuitionistic calculus is a conjunction calculus $D$ where there is binary an operator $\to: Ob(D)\times Ob(D) \to Ob(D)$ such that:
  \begin{enumerate}
  \item For each $A,B \in Ob(D)$, there is an arrow $\varepsilon^{A,B}: [(B \to A) \land B]\to A$.
  \item For each $g:C\land B \to A$ then there is an arrow $g^*: C\to [B\to A]$.
  \end{enumerate}
\end{definition}
\begin{remark}
  Note that we use $[]$ to avoid ambiguity between $\to$ as operator or as simply an arrow.
\end{remark}
Similarly to $\pi_i$, we will usually denote just $\varepsilon$. 
\begin{definition}
  A \emph{intuitionistic propositional calculus} is a positive intuitionistic calculus where there is a specified object $\bot$ and an operation $\lor: Ob(D)\times Ob(D) \to Ob(D)$ such that:
  \begin{enumerate}
  \item For each $A\in Ob(D)$ there is an arrow $\square_A: \Bot\to A$.
  \item For each $A_1,A_2\in Ob(D)$ there are arrows $\iin_i: A_i\to A_1 \lor A_2 $.
  \item For each $A,B,C \in Ob(D)$ there is an arrow $$h:[(B\to A)\land (C\to A)] \to [(C\lor B)\to A].$$
  \end{enumerate}
\end{definition}

\subsubsection{Categorical systems as Deduction systems}

We have already note how a category is a particular case of a deduction system. We can consider how are the logical systems, but we only select those who are also categories.\\

\begin{remark} {\color{red} Define $\langle f_1,f_2\rangle$ and $[f_1,f_2]$}
  
\end{remark}


\begin{proposition}
  A cartesian category $\mathcal{C}$ is a deduction system that is both conjunction system and a category where:
  \begin{enumerate}
  \item $f:A\to T\implies f=O_A$.
  \item Given $\langle f_1,f_2\rangle: C\to A\land B$ then $\pi_i \circ \langle f_1,f_2\rangle = f_i$.
  \item Given $h:C\to A\land B$ we have that $\langle \pi_1 h, \pi_2 h\rangle = h$.
  \end{enumerate}
\end{proposition}

\begin{proof}

  Lets begin by proving that a cartesian category satisfy this conditions. It is clearly a category. Using the specified terminal point $T=\Top$.  Considering the product $A\times B = A\land B$ for every $A,B \in C$ along with the projections, the others properties are satisfied.\\

  Conversely, considering again $T= \top$ we can check that it is a terminal object. Also, for every $h:C\to A\land B$we can see that every $h$ derive the existence of unique $f_1,f_2$ such that:
  \[
\begin{tikzcd}
{} & C \arrow[bend right,swap,dashed]{dl}{g}
\arrow[bend left,dashed]{dr}{h} \arrow{d}[description]{f}& & \\
A  &A\land B \arrow{l}[swap]{\pi_1} \arrow{r}{\pi_2} & 
B \\
\end{tikzcd}
\]
 And thus we have finite product.
\end{proof}

Thus, we can start seeing the kind of bijection already shown in the Curry Howard Isomorphism. Hitherto we can adopt the notation $A\times B$ for the conjunction, in the context of cartesian categories.
\begin{proposition}\label{def2:CCC}
  A closed cartesian category $\mathcal{C}$ is a deduction system that is a positive intuitionistic calculus that is also a category such that:
  \begin{enumerate}
  \item   For all $h: C\times B \to A$ , we have that $\varepsilon \langle h^* \pi_1, \pi_2\rangle = h$.
  \item   For all $k: C\to [B \to A]$ , we have that $(\varepsilon \langle k \pi_1, \pi_2\rangle)^* = k$.
  \end{enumerate}
\end{proposition}

\begin{proof}
  
  Lets begin checking that $\mathcal{C}$ is a positive intuitionistic calculus and holds 1. and 2. As it is a cartesian category, by it does satisfy the requirements for being a conjunction calculus. Pairing the exponential object $cA^B$ with the function object $B\to A$. As in definition \ref{def:CCC}, we have  3 adjoints, the first two of them are already considered as we have that $C$ is a cartesian category. We now study the third of them and remember that $F_3^B=F$ and $G_3^B=G$ are both adjoint, i.e.:
  $$\hom_C(C \times B, A) \equiv \hom(C, A^B)$$

  Therefore, we can check the second part in definition \ref{def:positivecalculus}. Then, remembering proposition \ref{prop:univAdjoint}, we can deduce the existence of a natural transformation  $\varepsilon: FG \to I_C$ i.e. $\varepsilon: A^B\times A = (B\to A)\land A \to A$, having the first part in definition \ref{def:positivecalculus}.\\

  The proof of 1. and 2. are conditions of universality. Lets unpack 1. as 2. is the same reasoning on the other side of the adjoint. We have (remember that, in the notation of part I: $\langle h^*\pi_1, \pi_2\rangle = \langle h^*,1_B\rangle$, and we used to note $h^*=\varphi(h)$):
\[
\begin{tikzcd}
  C\times B\arrow{d}{h}&C\arrow{d}{h^*} & C\times B\arrow{d}{\langle h^*,1_B\rangle} & A^B\times B\arrow{d}{\varepsilon}\\
  A & A^B = B\to A & A^B\times B & A 
\end{tikzcd}
\]

therefore, putting it all together, and remembering the \emph{evaluation} fashion of $\varepsilon$ (i.e., we have that the following diagram commutes. The 

\[
\begin{tikzcd}
  C\times B\arrow{d}{h} \arrow{r}{\langle h^*,1_B\rangle}& A^B\times B \arrow[swap]{dl}{\varepsilon}\\
  A&
\end{tikzcd}
\]

  
  To do the converse, is enough to made the same pairings and apply the notions of universality of $\varepsilon$ in  and apply \ref{prop:equivdefinition} to check that such adjoints are define via $\epsilon$.
\end{proof}

These propositions allow us to easily define graph completions to generate a to generate from these CCCs and positive conjunction calculus.

\begin{proposition} A \emph{closed bicartesian category} is a deduction system that is both a intuitionistic propositional calculus and a category, with the additional equations:
  \begin{enumerate}
  \item $f:\bot \to A \implies f=\square_A$.
  \item Given $[f_1,f_2] : A\lor B \to C$ then $ [f_1,f_2]\circ \iin_i  = f_i$.
  \item Given $h:A\lor B\to C$ we have that $[ h\iin_1, h\iin_2] = h$.
  \end{enumerate}
\end{proposition}
\begin{remark}
  Is usual to denote $A\lor B$ as $A+B$ or $A\sqcup B$.
\end{remark} 
\begin{proof}
  We have proved that it is already a closed cartesian category. Extra considerations are dual to does already considered for a cartesian category.
\end{proof}
\subsection{Adjoint of hypotheses}

We will now proceed to talk about unfinished products. This is as simple as asking: what would happen if this new truth existed? Or to put it another way in our new language, what if this new arrow were embedded in our graph. We will now study these inclusions and their consequences.\\

To do this, we are going to use the properties of the adjoint that is generated between the forgetful functor $\mathcal U$ and the free category $\mathcal{F}$ functor.{\color {red} add example y chapter 2}.\\


\begin{definition}[Category of closed cartesian categories]
\end{defintion}
\begin{proposition}
Given a cartesian closed category A and any graph morpshim $F: G \to \mathcal U(A)$ there is a unique morsphism in $Cart$  
\end{proposition}


\subsection{Natural Numbers}


Lambda calculus


\subsection{A language from a Closed Cartesian Category}


\subsubsection{The internal Language}
Explain Example 10.6 and proposition 10.7
\subsubsection{Equivalence}
After defining $L$ we want to show that it is an equivalence of categories. We shall obtain a functor $C$ in the opposite direction.  

% \subsubsection{Deduction Theorem}
% We proceed with the tating that an arrow $f:\top \to A$ does exists in our deduction system. It can be deduced in both positive (without $\lor$ and $\bot$) an intuitionistic calculus that is $A \land B\vdash C$ then $A \vdash C\to B$. This theorem is more interesting to state in the new optics of deduction systems:
% \begin{theorem}[Proposition 2.1, \cite{lambek1988introduction}]
%   In a positive  calculus, if assuming the existence of an arrow $f:\top \to A$ implies the existence of an arrow $g: B\to C$, then there exists an arrow $h: A\land B\to C$ that does not depend on $f$.
% \end{theorem}

% \begin{sproof}
% In a similar fashion of the Church-Rosser Theorem, it is solved by induction in the last rule used in the deduction.
% \end{sproof}


% As we have seen, a category is a deductive system with added structure.


\subsection{A review of Curry Howard under Deduction systems}